\documentclass[journal,twoside,web]{ieeecolor}
\usepackage{generic}
\usepackage{cite}
\usepackage{amsmath,amssymb,amsfonts}
\usepackage{algorithmic}
\usepackage{graphicx}
\usepackage{textcomp}
\def\BibTeX{{\rm B\kern-.05em{\sc i\kern-.025em b}\kern-.08em
    T\kern-.1667em\lower.7ex\hbox{E}\kern-.125emX}}
\markboth{\journalname, VOL. XX, NO. XX, XXXX 2017}
{Author \MakeLowercase{\textit{et al.}}: Preparation of Papers for IEEE TRANSACTIONS and JOURNALS (February 2017)}
\begin{document}
\title{Paper about Trajectory Control for IEEE TRANSACTIONS and JOURNALS}
\author{Francisco José Mañas-Álvarez, María Guinaldo, Raquel Dormido, Rafael Socas and Sebastián Dormido
\thanks{This paragraph of the first footnote will contain the date on 
which you submitted your paper for review. It will also contain support 
information, including sponsor and financial support acknowledgment. For 
example, ``This work was supported in part by the U.S. Department of 
Commerce under Grant BS123456.'' }
\thanks{The next few paragraphs should contain 
the authors' current affiliations, including current address and e-mail. For 
example, F. A. Author is with the National Institute of Standards and 
Technology, Boulder, CO 80305 USA (e-mail: author@boulder.nist.gov). }
\thanks{S. B. Author, Jr., was with Rice University, Houston, TX 77005 USA. He is 
now with the Department of Physics, Colorado State University, Fort Collins, 
CO 80523 USA (e-mail: author@lamar.colostate.edu).}
\thanks{T. C. Author is with 
the Electrical Engineering Department, University of Colorado, Boulder, CO 
80309 USA, on leave from the National Research Institute for Metals, 
Tsukuba, Japan (e-mail: author@nrim.go.jp).}}

\maketitle

\begin{abstract}
These instructions give you guidelines for preparing papers for 
IEEE Transactions and Journals. Use this document as a template if you are 
using \LaTeX. Otherwise, use this document as an 
instruction set. The electronic file of your paper will be formatted further 
at IEEE. Paper titles should be written in uppercase and lowercase letters, 
not all uppercase. Avoid writing long formulas with subscripts in the title; 
short formulas that identify the elements are fine (e.g., "Nd--Fe--B"). Do 
not write ``(Invited)'' in the title. Full names of authors are preferred in 
the author field, but are not required. Put a space between authors' 
initials. The abstract must be a concise yet comprehensive reflection of 
what is in your article. In particular, the abstract must be self-contained, 
without abbreviations, footnotes, or references. It should be a microcosm of 
the full article. The abstract must be between 150--250 words. Be sure that 
you adhere to these limits; otherwise, you will need to edit your abstract 
accordingly. The abstract must be written as one paragraph, and should not 
contain displayed mathematical equations or tabular material. The abstract 
should include three or four different keywords or phrases, as this will 
help readers to find it. It is important to avoid over-repetition of such 
phrases as this can result in a page being rejected by search engines. 
Ensure that your abstract reads well and is grammatically correct.
\end{abstract}

\begin{IEEEkeywords}
Control engineering, Motion planning, Multi-robot systems, Robot control
\end{IEEEkeywords}

\section{Introduction}
\label{sec:introduction}
\IEEEPARstart{T}{his} document is a template for \LaTeX. 


\subsection{Subsection}
 
\begin{equation}E=mc^2.\label{eq}\end{equation}


\smallskip\noindent
\begin{small}
\begin{tabular}{l}
\verb+\+\texttt{documentclass[journal,twoside,web]\{ieeecolor\}}\\
\verb+\+\texttt{usepackage\{\textit{Journal\_Name}\}}
\end{tabular}
\end{small}

\section{Second Section}


\section{Third Section}

\begin{figure}[!t]
\centerline{\includegraphics[width=\columnwidth]{fig1.png}}
\caption{Figure example. use of 
color/shades of gray}
\label{fig1}
\end{figure}

\section{Fourth Section}

\begin{table}
\caption{Table example}
\label{table}
\setlength{\tabcolsep}{3pt}
\begin{tabular}{|p{25pt}|p{75pt}|p{115pt}|}
\hline
Symbol& 
Quantity& 
Conversion from Gaussian and \par CGS EMU to SI $^{\mathrm{a}}$ \\
\hline
$\Phi $& 
magnetic flux& 
1 Mx $\to  10^{-8}$ Wb $= 10^{-8}$ V$\cdot $s \\
$B$& 
magnetic flux density, \par magnetic induction& 
1 G $\to  10^{-4}$ T $= 10^{-4}$ Wb/m$^{2}$ \\
$H$& 
magnetic field strength& 
1 Oe $\to  10^{3}/(4\pi )$ A/m \\
\hline
\multicolumn{3}{p{251pt}}{Table example footnote. }
\end{tabular}
\label{tab1}
\end{table}


\section{Conclusion}
A conclusion section is not required. Although a conclusion may review the 
main points of the paper, do not replicate the abstract as the conclusion. A 
conclusion might elaborate on the importance of the work or suggest 
applications and extensions. 

\appendices

\section*{Acknowledgment}

The preferred spelling of the word ``acknowledgment'' in American English is 
without an ``e'' after the ``g.'' Use the singular heading even if you have 
many acknowledgments. Avoid expressions such as ``One of us (S.B.A.) would 
like to thank $\ldots$ .'' Instead, write ``F. A. Author thanks $\ldots$ .'' In most 
cases, sponsor and financial support acknowledgments are placed in the 
unnumbered footnote on the first page, not here.



\begin{thebibliography}{00}

\bibitem{b1} G. O. Young, ``Synthetic structure of industrial plastics,'' in \emph{Plastics,} 2\textsuperscript{nd} ed., vol. 3, J. Peters, Ed. New York, NY, USA: McGraw-Hill, 1964, pp. 15--64.

\bibitem{b2} W.-K. Chen, \emph{Linear Networks and Systems.} Belmont, CA, USA: Wadsworth, 1993, pp. 123--135.


\end{thebibliography}

\begin{IEEEbiography}[{\includegraphics[width=1in,height=1.25in,clip,keepaspectratio]{a1.png}}]{First A. Author} (M'76--SM'81--F'87) and all authors may include 
biographies. Biographies are often not included in conference-related
papers.
\end{IEEEbiography}

\begin{IEEEbiography}[{\includegraphics[width=1in,height=1.25in,clip,keepaspectratio]{a2.png}}]{Second B. Author} was born in Greenwich Village, New York, NY, USA in 
1977. He received the B.S. and M.S. degrees in aerospace engineering from 
\end{IEEEbiography}

\begin{IEEEbiography}[{\includegraphics[width=1in,height=1.25in,clip,keepaspectratio]{a3.png}}]{Third C. Author, Jr.} (M'87) received the B.S. degree in mechanical 
engineering from 
\end{IEEEbiography}

\begin{IEEEbiography}[{\includegraphics[width=1in,height=1.25in,clip,keepaspectratio]{a4.png}}]{Fourth C. Author, Jr.} (M'87) received the B.S. degree in mechanical 
engineering from 
\end{IEEEbiography}

\begin{IEEEbiography}[{\includegraphics[width=1in,height=1.25in,clip,keepaspectratio]{a5.png}}]{Fourth C. Author, Jr.} (M'87) received the B.S. degree in mechanical 
engineering from 
\end{IEEEbiography}
\end{document}
